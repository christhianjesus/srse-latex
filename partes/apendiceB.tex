\chapter{Scripts SQL}
\label{ape:sql}

Antes de ejecutar los \textit{scripts}, para configurar \textit{PostgreSQL},
se recomienda seguir algunos de los pasos que se detallan en el siguiente enlace:

\url{https://www.postgresql.org/docs/current/non-durability.html}

En especial, se recomienda:

\begin{lstlisting}[caption={Configuración PostgreSQL.}]
synchronous_commit = off
fsync = off
full_page_writes = off
\end{lstlisting}

Además, se recomienda no agregar las claves primarias ni las foráneas,
se puede modificar la base de datos posteriormente para agregarlas.
Esto incluye, los índices que el sistema emplea.

Los \textit{scripts} se pueden encontrar en el repositorio siguiente:

\url{https://github.com/christhianjesus/srse-psql}

Estos \textit{scripts} incluyen:

\begin{itemize}
  \item Creación del esquema.
  \item Lectura de los archivos \textit{XML}.
  \item Extracción de los fragmentos de código.
  \item Creación del índice invertido (\textit{LSH}).
\end{itemize}