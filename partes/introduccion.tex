\chapter*{Introducción}

Desarrollar software es una actividad en constante evolución,
que con frecuencia obliga al programador a enfrentarse a nuevos desafíos,
debido a la creciente complejidad de los sistemas modernos.

Gran parte de estos desafios consiste en solucionar errores inesperados,
que surgen espontáneamente en la mayoría de los proyectos de software.
Estos fallos, denominados \textit{bugs}, se originan debido a errores de lógica o comprensión,
e incluso por comportamientos no definidos o inesperados~\cite{monperrus:hal-00987395}.
Su solución se dificulta al toparse con documentación escasa, incomprensible o inexistente.

Estudios muestran que el programador invierte hasta un 19\% de su tiempo~\cite{Brandt:2009:TSO:1518701.1518944}
consultando diversas fuentes~\cite{Ko:2007:INC:1248820.1248867},
entre las que destacan los recursos en línea~\cite{10.1007/978-0-387-09684-1_21}.

Los recursos en línea disponibles son variados y amplios,
siendo notables: Los foros, los blogs, las listas de correo electrónico, entre otros.
En particular, los sitios de preguntas y respuestas \ac{QA},
como \textit{Stack Overflow}, cuyos tópicos se enfocan en 
la resolución de problemas de programación
y conforman un conjunto aprovechable de conocimientos de libre acceso.

Sin embargo, surgen problemas inherentes al proceso de búsqueda,
entre los cuales destaca la falta de automatización~\cite{Ponzanelli:2014:PSR:2705615.2706035}.
Saber qué y cómo buscar información no es trivial,
el programador debe tener cierto grado de habilidad para realizar
búsquedas cuyos resultados sean satisfactorios~\cite{Stolee:2014:SSS:2628068.2581377}.

Por otro lado, las consultas hechas por los programadores
suelen emplear palabras clave que no se relacionan directamente con el código fuente.
El motivo aparente es la ineficacia de algunos motores de búsqueda para realizar este tipo de consultas,
un caso notable es el motor de búsqueda del sitio web de \textit{Stack Overflow}~\cite{monperrus:hal-00987395}.

\section*{Planteamiento del Problema}
La corrección de errores en el software es un proceso repetitivo y sistemático,
que puede llegar a ser tedioso y frustrante para el programador.
En general, tratar de resolver errores requiere consultar múltiples fuentes con frecuencia,
para buscar información que ayude a solucionar el error,
existiendo la posibilidad de no encontrar información adecuada.

Al programador no le es posible utilizar los motores de búsqueda para consultar el código directamente,
sino que, por el contrario, tiende a describir su problema brevemente enfatizando los puntos clave;
entre ellos, se incluye el nombre de la función, la clase, la librería o paquete,
los resultados obtenidos por el program, los errores indicados por el compilador,
la funcionalidad implementada, entre otros. En cualquier caso,
debe ser lo suficientemente perspicaz para describir correctamente el problema e identificar su origen.

Usualmente el programador tiende a consultar la documentación en busca de la solución al problema,
solo para encontrarse con que la documentación no abarca los conceptos necesarios para su resolución;
también, es posible que la misma sea de difícil comprensión para el programador,
al estar escrita de forma confusa, con términos complejos o en otro idioma;
además, cabe la posibilidad de que esté desactualizada y 
las características asociadas al problema estén relacionadas con una nueva implementación,
aún sin documentar o en el peor de los casos, no exista.
Lo que con seguridad impide al programador encontrar la solución al problema en periodos razonables de tiempo;
en el peor de los casos, puede obstaculizar y posponer la realización de la tarea asignada.

El programador que no ha encontrado la solución en la documentación,
con frecuencia recurre a otros sitios, entre ellos se encuentran los sitios de preguntas y respuestas.
Estos sitios permiten al usuario consultar a otros usuarios sobre el problema presentado,
permitiendo que las personas que hayan pasado por una situación similar
o con un conocimiento más amplio, respondan.
Estas respuestas quedan registradas en el foro y sirven como retroalimentación,
para el uso posterior por parte de otros usuarios.

Sin embargo, éste proceso es lento y requiere la intervención de otros usuarios.
Además, parte de las preguntas hechas por los usuarios se repite periódicamente,
a pesar que se enfatiza e insta a los usuarios a realizar una búsqueda anticipada.
Lo que se debe en esencia, a la falta de habilidad de algunos usuarios para realizar
consultas cuyos resultados sean satisfactorios.

\section*{Antecedentes}

Trabajos previos en el área apuntan a desarrollar un Sistema de Recomendación
que se beneficie del conocimiento provisto por la comunidad de \textit{Stack Overflow},
cuya principal motivación es proveer herramientas al usuario que agilicen el proceso de búsqueda.

Se han publicado diversas investigaciones en éste sentido, entre las que destacan:

\begin{itemize}
	\item \textbf{Debugging with the Crowd}\cite{monperrus:hal-00987395}
	presenta un Sistema de Recomendación como una herramienta para \textit{debugging}.
	Realiza un estudio empírico sobre la base de datos de Stack Overflow y establece las bases
	para el desarrollo de un sistema de recomendación que emplee los conocimientos de la comunidad de usuarios.
	
	\item \textbf{Prompter}\cite{Ponzanelli:2014:PSR:2705615.2706035} es un Sistema de Recomendación
	basado en \textit{Stack Overflow} que se muestra como un \textit{plug-in} para \textit{Eclipse},
	presenta un modelo de clasificación basado en las características del código y el contexto en
	el \textit{IDE}.

	\item \textbf{PDE4Java}\cite{Jadalla:2008:PPD:1413814.1413815} es un motor 
	para la detección de plagio en código fuente para \textit{Java}, indaga en
	las bases del uso del \textit{MinHash} como una potente herramienta.
\end{itemize}

Gran parte de la literatura relacionada, se enfoca en el estudio de distintas técnicas
que abordan el problema de encontrar similitudes en el código.

\section*{Justificación}

El presente Proyecto de Grado se aboca en el desarrollo e implementación de 
un Sistema de Recomendación, con la finalidad de proveer sugerencias 
relevantes al programador.

Muchos de los problemas que afectan el desarrollo de software se relacionan
directamente con el proceso de búsqueda de información. El programador
invierte tiempo valioso y sus consultas pueden no ser exitosas.
Incrementando la brecha entre los objetivos y el resultado.

Queda en evidencia la falta de herramientas
que faciliten el proceso de búsqueda al desarrollador,
permitan realizar consultas relacionadas con el código
y disminuyan el tiempo empleado.

Parte del enfoque de este Proyecto consiste en aplicar un algoritmo
eficaz que permita reconocer fragmentos de código estructuralmente similares,
salvando la gran cantidad de diferencias que pueden presentarse.

\section*{Objetivos}

\subsection*{Objetivo General}

%Desarrollar un Sistema de Recomendación capaz de sugerir tópicos en el sitio web
%de \textit{Stack Overflow} que sean de relevancia para el programador,
%ayuden a solucionar errores y agilicen el proceso de desarrollo de software
%en el lenguaje c/c++, apoyándose sobre algoritmos de Minería de Datos.

Desarrollar un sistema capaz de recomendar soluciones a errores cometidos por el programador en lenguaje C,
apoyándose sobre algoritmos de Minería de Datos.

\subsection*{Objetivos Específicos}
\begin{itemize}
  \item Aplicar algoritmos de minería de datos para explotar los conocimientos
  de programación en lenguaje c de la comunidad de programadores del sitio web \textit{Stack Overflow}.
  
  \item Implementar una herramienta que ayude al programador a identificar errores en el código,
  mediante una lista de recomendaciones de posibles soluciones proporcionada por la comunidad.

  \item Establecer un modelo de clasificación que incluya aspectos relacionados
  a la reputación de los usuarios de \textit{Stack Overflow}.
\end{itemize}

\section*{Organización del Libro}
El presente Proyecto de Grado está estructurado en cuatro capítulos.
En el Capítulo 1, el marco teórico,
se presentan los planteamientos necesarios para la comprensión del desarrollo del Sistema de Recomendación.
En el Capítulo 2, el marco tecnológico,
se exponen las herramientas empleadas que permitieron la implementación del Sistema de Recomendación.
En el Capítulo 3, el marco metodológico,
se desglosan los procesos, métodos y técnicas aplicadas para la elaboración del Sistema de Recomendación.
En el Capítulo 4, desarrollo y resultados,
se describen los resultados obtenidos al emplear el índice invertido e implementar el Sistema de Recomendación.
