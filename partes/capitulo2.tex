\chapter{Marco Tecnológico}

En este capítulo se explica brevemente cada uno de los componentes necesarios
para el desarrollo del Sistema de Recomendación del presente Proyecto de Grado.
En primer lugar, en la Sección \ref{sec:ide}, se identifica el entorno de desarrollo integrado \textit{Eclipse}.
En segundo lugar, en la Sección \ref{sec:languajes}, se describen los lenguajes \textit{Java} y \textit{PL/pgSQL}.
Posteriormente, en la Sección \ref{sec:libs}, se enuncian las librerías empleadas por el sistema.
Finalmente, en la Sección \ref{sec:db}, se exponen las fortalezas del uso de \textit{PostgreSQL}.

\section{Entorno de Desarrollo Integrado}
\label{sec:ide}

Se desarrolla el Sistema de Recomendación como un \textit{plug-in}
de \textit{Eclipse} por su gran flexibilidad.
Además, este \ac{IDE} o entorno de desarrollo integrado,
incluye un \textit{parser} para el lenguaje c y c++,
que está escrito en \textit{Java}.

\subsection{Eclipse}
\textit{Eclipse} es una plataforma de software escrita en \textit{Java} en su mayoría,
que integra un conjunto de herramientas de programación de código abierto multiplataforma
y es usado típicamente como un \ac{IDE}, siendo algunas de sus extensiones \ac{JDT} y \ac{CDT}.
Contiene un espacio de trabajo base y un sistema de complementos extensible,
que permite personalizar el entorno de desarrollo.

\section{Lenguajes de Programación}
\label{sec:languajes}

Los lenguajes de programación \textit{Java} y \textit{PL/pgSQL}
son usados para desarrollar complementos en \textit{Eclipse} y
\textit{scripts} en \textit{PostgreSQL}, respectivamente.
%Se emplean ambos lenguajes para la elaboración del Sistema de Recomendación.

\subsection{Java}

\textit{Java} es un lenguaje de programación orientado a objetos de propósito general,
que fue creado en 1991 por James Gosling y publicado en 1995 por \textit{Sun Microsystem}.
Su sintaxis deriva en gran medida de los lenguajes c y c++,
pero tiene menos control de las estructuras a bajo nivel que cualquiera de ellos.

%El lenguaje\textit{Java} fue diseñado pensando en la portabilidad, su intención es permitir que los programadores escriban el programa una sola vez y lo ejecuten en cualquier dispositivo. Para lograr este propósito, se genera una representación intermedia llamada \textit{bytecode} cuando se compila una aplicación \textit{Java}, que se puede ejecutar en cualquier \textit{Máquina Virtual Java} sin importar la arquitectura de la plataforma subyacente.

\subsection{PL/pgSQL}

\ac{PL/pgSQL},
es un lenguaje imperativo provisto por el gestor de base de datos \textit{PostgreSQL},
escrito originalmente por Jan Wieck.

Los objetivos de diseño de \ac{PL/pgSQL} se enfocaron en crear un lenguaje procedural
cargable que pueda usarse para crear funciones y procedimientos disparados por eventos,
añada estructuras de control al lenguaje \ac{SQL}, pueda realizar cálculos complejos,
herede todos los tipos definidos por el usuario, las funciones y los operadores,
pueda ser definido para ser confiable para el servidor y sea fácil de usar.

%La principal ventaja es, que permite minimizar el número consultas al servidor de base de datos que dependen de un cálculo complejo, lo cual conlleva a un ahorro sustancial de recursos. Además, permite al gestor de base de datos planificar optimizaciones en la ejecución de la búsqueda y actualización de los datos.

\section{Librerías}
\label{sec:libs}

El Sistema de Recomendación utiliza
\textit{Google Guava} para extender las funcionalidades básicas de \textit{Java},
\textit{Apache Commons Text} para decodificar entidades especiales en \ac{XML} y \ac{HTML},
la librería \textit{CORE} de \textit{Eclipse CDT} para realizar el análisis léxico en c y c++.

\subsection{Google Guava}

\textit{Google Guava} es un conjunto de librerías comunes de código abierto para \textit{Java},
desarrollado principalmente por \textit{Google}. Entre sus características incluye: 

\begin{itemize}
  \item Tipos para colecciones.
  \item Colecciones inmutables.
  \item Biblioteca de gráficos.
  \item Tipos funcionales.  
  \item Cache en memoria.  
  \item Utilidades para concurrencia.
  \item Operaciones de entrada y salida (I/O).  
  \item Hashing.
  \item Primitivas.
  \item Reflexión.
  \item Procesamiento de cadenas de carácteres.
\end{itemize}

\subsection{Apache Commons Text}
\label{subsec:apache}

\textit{Apache Commons Text} forma parte de un conjunto de proyectos de \textit{Apache Software Foundation}.
El propósito de estos proyectos consiste en proveer componentes de software \textit{Java} reutilizables, en código abierto.
Ésta librería es una biblioteca de métodos enfocada en algoritmos para procesar cadenas de carácteres.

\subsection{Eclipse CDT: CORE}

La librería \textit{CORE} de \textit{Eclipse CDT} contiene 
dos \textit{parsers} de código libre escritos en \textit{Java},
para c y c++, que son conocidos como los \textit{DOM Parsers}.
Como \textit{Eclipse CDT} no es un compilador, sino un \ac{IDE},
el diseño de los \textit{parsers} tiene otro enfoque,
cuya principal influencia es el rendimiento.

Entre sus principales características se encuentra:

\begin{itemize}
  \item Soporte para c y c++ actualizado.
  \item Soporta recuperación de errores.
  \item Incluye preprocesador.
  \item Se encuentra en constante mantenimiento.
\end{itemize}

Aunque la documentación para el uso del \textit{parser} de forma independiente es prácticamente inexistente,
existen trabajos de investigación en el área~\cite{10.1007/978-3-642-33442-9_45}.

\section{Bases de Datos}
\label{sec:db}

Las bases de datos constituyen una de las fuentes de conocimiento de un \textit{Sistema de Recomendación}.
Se utilizó \textit{PostgreSQL} como manejador de bases de datos por su versatilidad,
flexibilidad y robustez al tratar con una gran cantidad de información.

\subsection{PostgreSQL}

\textit{PostgreSQL} es un \ac{RDBMS}, basado en \textit{POSTGRES v4.2},
desarrollado en el Departamento de Ciencias de la Computación de la
Universidad de California en Berkeley.

\textit{PostgreSQL} es un descendiente código abierto del código original
desarrollado en Berkeley y compatible con una gran parte del estándar \ac{SQL}.
Además, ofrece características modernas como consultas complejas,
claves foráneas, disparadores, vistas actualizables, integridad
transaccional y control de concurrencia multiversión.

Además, \textit{PostgreSQL} es flexible y puede ser extendido por el usuario para agregar:

\begin{itemize}
  \item Tipos de datos.
  \item Funciones.
  \item Operadores.
  \item Funciones de agregación.
  \item Métodos de Indexación.
  \item Lenguajes Procedurales.
\end{itemize}

%Adicionalmente, debido a la licencia permisiva, \textit{PostgreSQL} puede ser utilizado, modificado y distribuido por cualquier persona de forma gratuita para cualquier propósito, ya sea privado, comercial o académico.