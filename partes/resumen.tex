\begin{resumen}
En los \'ultimos a\~nos ha crecido la popularidad de los sitios de preguntas y respuestas como un recurso informativo,
capaz de proveer a sus usuarios de un medio para obtener soluciones a diversos problemas.
Se sabe que los programadores tienden a invertir gran parte de su tiempo buscando soluciones a errores en estos sitios.
%invirtiendo incluso un 19\% del tiempo que disponen.

Asimismo, estudios previos han demostrado la baja efectividad de ciertos
motores de b\'usqueda al emplear c\'odigo en sus consultas, entre ellos Stack Overflow.
Lo que fuerza al programador a intentar describir su problema en breves l\'ineas
y dificulta la b\'usqueda de informaci\'on. 

Surge la necesidad de desarrollar una herramienta que automatice el proceso,
por tal motivo se desarroll\'o un Sistema de Recomendaci\'on para ayudar al programador a identificar errores mediante sugerencias,
que emplea fragmentos de c\'odigo para la consulta y 
aplica algoritmos de Miner\'ia de Datos para explotar los conocimientos de la comunidad de usuarios de Stack Overflow.

Se compararon las fortalezas y debilidades de las t\'ecnicas de similitud entre documentos
y se decidi\'o usar una modificaci\'on del esquema de Document Fingerprinting,
que emplea la t\'ecnica de MinHash para realizar comparaciones entre fragmentos de c\'odigo
y utiliza la t\'ecnica de Locality-Sensitive Hashing para indexar los fragmentos.

Las pruebas realizadas sobre el \'indice mostraron una efectividad del 100\%
sobre una muestra de $1.000$ fragmentos de c\'odigo extra\'idos de los t\'opicos.
Al emplear otra muestra de 10 fragmentos extra\'idos de fuentes externas mostr\'o ser eficaz al 60\%,
sugiriendo t\'opicos relacionados. Se concluye entonces que el sistema
implementado cumple cabalmente la funci\'on para la cual fue dise\~nado.

\vfill
\textbf{Palabras clave:} MinHash, LSH, Sistema de Recomendación, Stack Overflow, Plugin de Eclipse.
\end{resumen}