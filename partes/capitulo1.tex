\chapter{Marco Teórico}

En este capítulo se definen los conceptos necesarios para la comprensión
del Sistema de Recomendación del presente Proyecto de Grado.
En primer lugar, en la Sección \ref{sec:min}, se provee una breve introducción a Minería de Datos.
En segundo lugar, en la Sección \ref{sec:sr}, se presentan los fundamentos de un Sistema de Recomendación.
En tercer lugar, en la Sección \ref{sec:qa}, se describen los Sitios de Preguntas y Respuestas.
En cuarto lugar, en la Sección \ref{sec:sim}, se expone la Similitud entre Documentos como un área de investigación.
Posteriormente, en la Sección \ref{sec:doc}, se detalla la técnica de \textit{Document Fingerprinting} empleada.
Finalmente, en la Sección \ref{sec:ind}, se enuncian las bases para la elaboración del Índice Invertido.

\section{Minería de datos}
\label{sec:min}

“La Minería de Datos \dots~implica la deducción de algoritmos que exploran los datos,
desarrollan el modelo y descubren patrones previamente desconocidos”~\cite{Maimon:2010:DMK:1869923}.

El objetivo general del proceso de minería de datos consiste en extraer
información de una fuente de datos e intentar convertir esa información
en un conjunto aprovechable de conocimientos.

El proceso de minería de datos generalmente consiste de 3 pasos~\cite{Amatriain2011}:

\begin{itemize}
  \item Preprocesamiento.
  \item Análisis.
  \item Interpretación.
\end{itemize}

Los Sistemas de Recomendación nacen como consecuencia de automatizar el
proceso de una técnica de minería de datos, para una instancia en particular.
 
\section{Sistema de Recomendación}
\label{sec:sr}

Los Sistemas de Recomendación son agentes de software que obtienen los intereses
o preferencias de los usuarios, para hacer las recomendaciones correspondientes~\cite{Xiao:2007:EPR:2017327.2017335}.
Generalmente, éstas sugerencias se generan utilizando información perteneciente al usuario,
ya sea que esta información fuere adquirida de forma explícita o implícita.

\subsection{Objetivos}

El objetivo principal de un Sistema de Recomendación es realizar sugerencias 
que permitan agilizar un proceso.

Además, se mencionan los siguientes objetivos operacionales y técnicos~\cite{Aggarwal2016}:

\begin{description}
  \item [Relevancia] Recomendar elementos que sean relevantes para el usuario es el objetivo operativo esencial de un Sistema de Recomendación.
  \item [Novedad] Si el contenido del tópico recomendado es diferente a los que ha visto el usuario en el pasado pero sigue siendo útil.
  \item [Serendipia] Los elementos recomendados son algo inesperados y por lo tanto, hay un modesto sentimiento de descubrimiento afortunado, en contraposición a las recomendaciones obvias.
\end{description}

\subsection{Modelos Básicos}

Se definen tres modelos básicos de Sistemas de Recomendación~\cite{jannach_zanker_felfernig_friedrich_2010}:

\subsubsection{Filtrado Colaborativo}

Los Sistemas de Recomendación que utilizan modelos de filtrado colaborativos
realizan sus recomendaciones empleando las calificaciones proporcionadas por múltiples usuarios.
La idea básica consiste en totalizar las calificaciones sobre los elementos,
reconocer aquellas calificaciones que son comunes entre los usuarios y
generar nuevas recomendaciones basadas en la comparación de estas calificaciones.
En otras palabras, aprovechar las correlaciones entre elementos o
las correlaciones entre los usuarios para el proceso de predicción.

\subsubsection{Basados en Contenido}

En los Sistemas de Recomendación basados en contenido,
los atributos descriptivos de los elementos se usan para hacer recomendaciones,
el término contenido se refiere a estos atributos.
Se establecen perfiles para los elementos,
se encuentran aquellos que comparten características similares y
se recomiendan a los usuarios en base a elecciones anteriores.

\subsubsection{Basados en Conocimiento}

En los Sistemas de Recomendación basados en conocimiento,
las recomendaciones se llevan a cabo en base a las similitudes
entre los requerimientos del usuario y las características de los elementos.
Durante el proceso de recuperación se utilizan diversas reglas y
funciones de similitud provenientes del uso de bases de conocimiento.
Este enfoque toma su nombre de las bases de conocimiento.

\subsubsection{Sistemas Híbridos}

Los modelos básicos de Sistemas de Recomendación, explotan diferentes fuentes de datos.
Los sistemas híbridos pueden combinar las fortalezas de varios tipos
de Sistemas de Recomendación para crear un modelo más robusto;
además de combinar múltiples fuentes de datos,
también pueden mejorar la eficiencia de una clase en particular.

\section{Sitio de Preguntas y Respuestas}
\label{sec:qa}

Un \ac{QA} o sitio de preguntas y respuestas,
permite a sus usuarios adquirir, compartir y generar
conocimiento al interactuar con la comunidad.
%Los usuarios pueden consultar, participar y contribuir
%activamente con la comunidad a través de internet.

Se compone de tres partes fundamentales:

\begin{description}
  \item [Preguntas] Las preguntas son formuladas por los usuarios para describir
  un problema con el fin de obtener una solución por parte de la comunidad.
  %Es necesario que el usuario recolecté toda la información necesaria al momento de formular la pregunta.
  \item [Respuestas] Las respuestas son suministradas por otros usuarios y %supondrán una contribución a la comunidad como 
  representan posibles soluciones al problema planteado en la pregunta asociada.
  %Es posible que varias de estas posibles soluciones puedan resolver el problema.
  \item [Comentarios] Los comentarios son utilizados por los usuarios para agilizar el proceso; 
  permiten aclarar detalles, sugerir correcciones o proveer información.
\end{description}

\subsection{Stack Overflow}

\textit{Stack Overflow} es un \ac{QA} perteneciente a la red \textit{Stack Exchange},
que se enfoca en la resolución de problemas de programación.

El contenido de los tópicos es provisto por la comunidad y
conforma un conjunto aprovechable de conocimientos de libre acceso.

\subsubsection{Estructura de la Base de Datos}
\label{subsec:EstBasDat}

La base de datos de \textit{Stack Overflow} se compone de 8 tablas distintas,
cuya descripción se presenta a continuación:

\begin{description}
  \item [Badges] Conjunto que comprende a las medallas otorgadas a cada usuario.
  Son premios por contribuir activamente con la comunidad.
  \item [Comments] Conjunto que comprende a los comentarios asociados a los tópicos.
  Permite agilizar el intercambio de información.
  \item [PostHistory] Conjunto que comprende cada una de las modificaciones realizadas a los tópicos.
  Un historial de cambios.
  \item [PostLinks] Conjunto que comprende a los enlaces entre tópicos,
  ya sean tópicos relacionados o tópicos duplicados.
  \item [Posts] Conjunto que comprende el contenido esencial del sitio, los tópicos;
  ya sean éstos preguntas o respuestas.
  \item [Tags] Conjunto que comprende a todas las etiquetas que han sido asociadas a los tópicos.
  Cada tópico puede tener hasta 5 etiquetas asociadas.
  \item [Users] Perfil del usuario, conjunto que comprende la información pública para cada usuario.
  \item [Votes] Reputación de los tópicos, conjunto que comprende cada voto asignado a los tópicos.
\end{description}

Se puede encontrar el diagrama \textit{entidad-relación} en el Anexo \ref{ape:diag},
así como una descripción detallada del contenido de las tablas.
%Además, se incluyen un enlace al diagrama general de la base
%de datos de los sitios web pertenecientes a la red \textit{Stack Exchange}.

\section{Similitud entre Documentos}
\label{sec:sim}

Existen diversas áreas de investigación que abordan el problema de encontrar
documentos similares, entre ellas se encuentran Detección de Plagio
y Detección de Clones.

\subsection{Detección de Plagio}
\label{subsec:plagio}

“Plagio, es el acto de tomar los escritos de otra persona y pasarlos como propios.
El fraude está estrechamente relacionado con la falsificación y la piratería,
prácticas generalmente en violación de las leyes de derechos de autor”~\cite{britannica_2017}.

Las características propias de los lenguajes de programación,
se presentan como dificultades para el análisis de similitud textual~\cite{Beth2014ACO},
permitiendo realizar una gran cantidad de cambios al código.

A continuación, se enumeran los modificaciones más comunes~\cite{Arwin:2006:PDA:1151699.1151730}:

\begin{itemize}
  \item cambio de formato
  \item cambio de identificadores
  \item cambio en el orden de los operandos en las expresiones
  \item cambio del tipo de datos
  \item sustitución de expresiones por sus equivalentes
  \item introducción de declaraciones redundantes
  \item cambio en el orden de declaraciones independientes del tiempo
  \item cambio en la estructura de las instrucciones de iteración
  \item cambio en la estructura de las declaraciones de selección
  \item sustitución de las llamadas a función por el cuerpo de la misma
  \item introducción de declaraciones no estructuradas
  \item combinación de fragmentos de código originales y copiados
  \item traducción del código fuente de un lenguaje a otro
\end{itemize}

\newpage
El plagio generalmente ocurre en dos niveles~\cite{Aggarwal2016}:

\begin{description}
  \item [Plagio Textual] El documento plagiado es idéntico o similar al documento original, el cual puede ser un informe, un ensayo, un reporte, entre otros.
  \item [Plagio de Código Fuente] El código plagiado es una copia completa o parcial del código original, que ha sido adulterado para pasarlo como propio.
\end{description}

\subsection{Detección de Clones}
\label{subsec:plagio}
Un clon es un fragmento de código que ha sido copiado,
aún si éste ha sufrido modificaciones menores
o ha sido adaptado a otro contexto.
Sin embargo, su definición es vaga y muchas veces depende de los algoritmos
y los ajustes utilizados~\cite{Roy07asurvey}.

Aunque los enfoques son distintos,
tanto el área de Detección de Clones,
como el área de Detección de Plagio en Código Fuente,
se fundamentan en determinar la similitud entre fragmentos de código arbitrario~\cite{Beth2014ACO}.

A continuación, se presenta una clasificación de los tipos de clones
basada en el tipo de similitud~\cite{Roy07asurvey}:

\begin{description}
  \item [Tipo I] Fragmentos de código idénticos exceptuando las variaciones en los espacios en blanco, el diseño y los comentarios.
  \item [Tipo II] Fragmentos de código sintácticamente idénticos exceptuando las variaciones en los identificadores, los literales, los tipos, el diseño y los comentarios.
  \item [Tipo III] Fragmentos de código con modificaciones adicionales. Las expresiones pueden estar alteradas y tener variaciones en los identificadores, los literales, los tipos, el diseño y los comentarios.
  \item [Tipo IV] Fragmentos de código que realizan el mismo cálculo pero implementados a través de diferentes variantes sintácticas.
\end{description}

\newpage
\subsection{Métodos y Técnicas}

Numerosos métodos y técnicas han sido desarrollados para abordar
el problema de similitud entre fragmentos de código.

Una clasificación general fundamentada en el tipo de comparación es provista~\cite{Roy07asurvey}:

\begin{description}
  \item [Comparación basada en texto] Busca coincidencias textuales exactas entre segmentos; por ejemplo, ejecuciones de cinco palabras. De ejección rápida, pero puede confundirse al renombrar identificadores.
  \item [Comparación basada en lexemas] Similar a la comparación textual, pero usando un \textit{lexer} para convertir el programa en \textit{tokens}. Se descartan los espacios en blanco, los comentarios y los nombres de los identificadores.
  \item [Comparación basada en árboles de sintaxis] Se construyen y comparan los árboles de sintaxis abstracta, permitiendo detectar similitudes de mayor nivel; por ejemplo, puede normalizar sentencias condicionales y detectar constructos equivalentes.
  \item [Comparación basada en grafos de dependencia] Captura el flujo de control en un programa y permite ubicar equivalencias de niveles mucho más altos, que involucra una mayor complejidad y tiempo de cálculo.
  \item [Comparación basada en métricas] Calculan puntajes para segmentos de código de acuerdo a ciertos criterios; por ejemplo, el número de ciclos y condicionales. Sin embargo, puede dar lugar a falsos positivos.
  \item [Comparación empleando enfoques híbridos] Por ejemplo, árboles de sintaxis abstracta con árboles de sufijo, puede combinar la capacidad de detección de árboles de sintaxis abstracta con la velocidad que permiten los árboles de sufijo.
\end{description}

\section{Document Fingerprinting}
\label{sec:doc}

Es una técnica para la detección precisa de copias completas 
y parciales entre documentos~\cite{Lindkvist2005WinnowingA}.
El texto se divide en subsecuencias, se calcula el \textit{hash} para cada 
subsecuencia y se comparan los valores de \textit{hash} obtenidos entre los documentos. 
Este proceso se realiza con la idea de que la reducción de la dimensionalidad resultante,
supondrá una comparación mucho más factible y conservará la misma relación de similitud.
En otras palabras, si un valor \textit{hash} es igual a otro valor \textit{hash}, 
es muy probable que los textos correspondientes también sean idénticos~\cite{Beth2014ACO}.

\subsection{Subsecuencias}

El texto se divide en subsecuencias mediante el uso de \textit{n-gram} o \textit{w-shingling},
donde el $n$ y el $w$ son parámetros escogidos por el usuario~\cite{Beth2014ACO}.

\subsubsection{N-Grams}

Un \textit{n-gram} es una subsecuencia de $n$ elementos contiguos,
creada a partir de una secuencia dada. Por ejemplo, 
la secuencia de \textit{4-grams} de la siguiente frase
‘pablito clavó un clavito’ es:

\begin{equation*}
pabl\ abli\ blit\ lito\ itoc\ toca\ ocav\ cavo\ avou\ voun\ ounc\ uncl\ ncla\ clav\ lavi\ avit\ vito
\end{equation*}

\subsubsection{W-Shingles}

Un \textit{shingle} es una subsecuencia contigua de palabras. 
Por su parte, una subsecuencia contigua de $w$ palabras es un \textit{w-shingle}.
Por ejemplo, el conjunto de \textit{2-shingles} de la frase ‘compré pocas copas, pocas copas compré’ es:

\begin{equation*}
(compre,\ pocas),\ (pocas,\ copas),\ (copas,\ pocas),\ (copas,\ compre)
\end{equation*}

\subsubsection{Hashing}

Las subsecuencias de texto obtenidas mediante \textit{n-grams}
o \textit{w-shingles} no suelen usarse directamente;
por el contrario, se escoge una función de \textit{hash} que convierte cada una de las
subsecuencias de texto a una representación numérica.

Por ejemplo, se podría escoger una representación de \textit{n-grams} de $n = 7$
para un documento; luego, se utiliza una función de \textit{hash} y 
se convierten los \textit{7-grams} a una representación numérica
en el rango de $0$ a $2^{32} - 1$. 
En consecuencia, cada \textit{7-gram} es representado por \textit{4 bytes} en vez de \textit{7 bytes},
lo que compacta el tamaño de los datos y permite realizar operaciones sobre enteros.

\subsection{Similitud}

%La medida de similitud entre documentos se puede derivar del número de las partes compartidas entre los documentos.
Si se ve al documento como un conjunto, los elementos del conjunto serán aquellos valores de \textit{hash} que son calculados a partir del documento.
De esta manera, los documentos que compartan fragmentos de texto similares tendrán elementos comunes en sus conjuntos,
incluso si estos fragmentos aparecen en diferente orden en los documentos~\cite{Rajaraman:2011:MMD:2124405}.

\subsubsection{Coeficiente de Similitud de Jaccard}

Se mide la similitud de conjuntos por el coeficiente de similitud de \textit{Jaccard},
el cual para los conjuntos $A$ y $B$ es~\cite{Rajaraman:2011:MMD:2124405}:

\begin{equation*}
J(A, B) = \frac{|A \cap B|}{|A \cup B|}
\end{equation*}

Esto es, la relación entre el tamaño de la intersección de $A$ y $B$ con el tamaño de su unión.
El valor obtenido se encuentra entre $0$ y $1$, 
siendo $1$ cuando los conjuntos son iguales y 
$0$ cuando son diferentes. 
Los valores intermedios definen el grado de similitud.

\subsection{Huella Digital}

La técnica de \textit{Document Fingerprinting} no se limita a 
comparar los valores de \textit{hash} obtenidos de los documentos directamente; 
en su lugar, se selecciona un subconjunto de estos valores de \textit{hash}
para crear el \textit{fingerprint} o huella digital.
De esta manera, se puede representar el documento en menos espacio
y realizar comparaciones mucho más rápidas.

\subsubsection{Firma}

La mayoría de las técnicas utilizadas en \textit{Document Fingerprinting} 
son diseñadas para detectar diferencias entre documentos,
generando valores de \textit{hash} que difieren significativamente si hay cambios.
En contraposición, se utiliza una firma para detectar similitudes entre documentos,
produciendo valores de \textit{hash} similares para documentos similares.

\subsubsection{MinHash}

El \textit{MinHash} está diseñado con el fin de que dos objetos
similares generen valores de \textit{hash} similares~\cite{grattan_2017}.
Se suele utilizar ésta representación para estimar de forma eficiente
el coeficiente de similitud de \textit{Jaccard}~\cite{Rajaraman:2011:MMD:2124405}.

Una forma de visualizar el funcionamiento del método
consiste en utilizar una matriz característica,
donde las columnas corresponden a los conjuntos y
las filas a sus elementos.
De esta forma, se obtiene una matriz que contiene un $1$ en la fila $f$ y la columna $c$,
si el elemento de la fila $f$ es miembro del conjunto de la columna $c$;
caso contrario, el valor en la posición $(f,c)$ es $0$.

Luego, al escoger una permutación de las filas de la matriz, 
se tiene que el valor \textit{MinHash} para el conjunto de la columna $c$,
es el valor del elemento de la primera fila $f$ en ese orden,
para el cual el valor en la posición $(f,c)$ es $1$.

Por ejemplo, si se tiene la matriz característica cuyas filas son los elementos
$\{a,b,c,d,e\}$ y sus columnas son los subconjuntos $S_1 = \{a,d\}$,
$S_2 = \{c\}$, $S_3 = \{b,d,e\}$ y $S_4 = \{a,c,d\}$, se obtendrá la Tabla \ref{tab:conjuntoejemplo1}.

\begin{table}[h]
\caption{Matriz característica.}
\label{tab:conjuntoejemplo1}
\small
\centering
\begin{tabular}{ccccc}
\hline
{Elemento} & {$S_1$} & {$S_2$} & {$S_3$} & {$S_4$} \\
\hline
a & 1 & 0 & 0 & 1 \\
b & 0 & 0 & 1 & 0 \\
c & 0 & 1 & 0 & 1 \\
d & 1 & 0 & 1 & 1 \\
e & 0 & 0 & 1 & 0 \\
\hline
\end{tabular}
\end{table}

Luego, al generar una permutación de las filas de la matriz en la Tabla \ref{tab:conjuntoejemplo1},
tal que el orden de las filas sea ‘beadc’, se obtendrá la matriz característica
que se muestra en la Tabla \ref{tab:conjuntoejemplo2}.

\begin{table}[h]
\caption{Matriz característica permutada.}
\label{tab:conjuntoejemplo2}
\small
\centering
\begin{tabular}{ccccc}
\hline
{Elemento} & {$S_1$} & {$S_2$} & {$S_3$} & {$S_4$} \\
\hline
b & 0 & 0 & 1 & 0 \\
e & 0 & 0 & 1 & 0 \\
a & 1 & 0 & 0 & 1 \\
d & 1 & 0 & 1 & 1 \\
c & 0 & 1 & 0 & 1 \\
\hline
\end{tabular}
\end{table}

Esta permutación define una función  $h_1(x)$,
que convierte los subconjuntos $S_1, S_2, \ldots, S_n$ a filas 
$h_1(S_1), h_1(S_2), \ldots, h_1(S_n)$.
De esta forma, se obtiene el valor de \textit{MinHash} para $S_1$ de acuerdo a $h_1(x)$.
Así se tiene que, el valor del elemento de la primera fila,
que contiene un $1$ para el conjunto de la columna $S_1$ es $a$,
luego $h_1(S_1) = a$. Se obtienen los valores para $S_2$, $S_3$ y $S_4$ de forma similar,
tal que $h_1(S_2) = c$, $h_1(S_3) = b$ y $h_1(S_4) = a$.

A continuación, se escogen $2$ permutaciones distintas de las filas,
tal que el orden de las filas sea ‘cdaeb’ y ‘dbcea’, lo que genera $h_2(x)$ y $h_3(x)$,
como se muestra en la Tabla \ref{tab:conjuntoejemplo3}.

\begin{table}[h]
\caption{Matriz característica con 2 permutaciones.}
\label{tab:conjuntoejemplo3}
\small
\centering
\begin{tabular}{ccccc}
\hline
{Elemento} & {$S_1$} & {$S_2$} & {$S_3$} & {$S_4$} \\
\hline
c & 0 & 1 & 0 & 1 \\
d & 1 & 0 & 1 & 1 \\
a & 1 & 0 & 0 & 1 \\
e & 0 & 0 & 1 & 0 \\
b & 0 & 0 & 1 & 0 \\
\hline
\end{tabular}
\quad
\begin{tabular}{ccccc}
\hline
{Elemento} & {$S_1$} & {$S_2$} & {$S_3$} & {$S_4$} \\
\hline
d & 1 & 0 & 1 & 1 \\
b & 0 & 0 & 1 & 0 \\
c & 0 & 1 & 0 & 1 \\
e & 0 & 0 & 1 & 0 \\
a & 1 & 0 & 0 & 1 \\
\hline
\end{tabular}
\end{table}

Estas funciones generan una secuencia de valores de tamaño $3$, 
que seran las firmas para los subconjuntos $S_1$, $S_2$, $S_3$ y $S_4$, tal que:
\begin{align*}
firma(S_1) = [a,d,d] \\
firma(S_2) = [c,c,c] \\
firma(S_3) = [b,d,d] \\
firma(S_4) = [a,c,d]
\end{align*}

Una vez generadas las firmas, si estas son de tamaño $n$ para dos conjuntos arbitrarios $A$ y $B$,
al compararlas, se tiene que cada valor obtenido $r_i$ se comporta como una variable aleatoria $0/1$,
que es igual a 1 solo cuando $h_i(A) = h_i(B)$, para todo $i$ en el rango $1<=i<=n$.
Entonces, $r_i$ es un estimador insesgado para $Jaccard(A, B)$, 
que se promedia con las $n-1$ variables aleatorias restantes para mitigar la varianza.

\begin{equation*}
MinHash(A, B) = \displaystyle\sum_{i=1}^{n} \frac{r_i}{n}
\end{equation*}

Esta técnica permite realizar comparaciones en tiempo de ejecución $O(n)$.

\newpage
Como $r_i$ es un estimador insesgado, su promedio también lo es y
por desviación estándar para sumas de variables aleatorias $0/1$~\cite{coms699812},
se tiene que el error esperado es:

\begin{equation*}
O(\frac{1}{\sqrt{k}})
\end{equation*}

Por ejemplo, si se comparan las firmas para $S_1$ y $S_4$, vemos:

\begin{equation*}
Jaccard(S_1, S_4) = MinHash(S_1, S_4) = 66.67\%
\end{equation*}

\subsubsection{Función de Hash Universal}
Se emplean una familia de funciones de \textit{hash} universal,
ya que no es óptimo computar las tablas y permutar sus elementos,
estas funciones son de la forma:

\begin{equation*}
h(x) = (a * x + b) \bmod c
\end{equation*}

Si $n$ es la cantidad de elementos en el conjunto, se tiene que:

\begin{itemize}
  \item $x$ es un valor que representa la posición del elemento en el conjunto.
  \item $a$ y $b$ son valores seleccionados al azar en el intervalo $1$ a $n$.
  \item $c$ es el siguiente número primo mayor o igual a $n$.
\end{itemize}

\section{Índice Invertido}
\label{sec:ind}
El índice invertido es una estructura de datos que suele asociar
texto a la localización de un documento en un conjunto de datos.

El propósito de un índice invertido es optimizar las búsquedas basadas en fragmentos de datos compuestos,
lo que requiere cierto preprocesamiento al agregar un nuevo elemento al conjunto de datos.

\newpage
\subsection{Locality-Sensitive Hashing}
El enfoque general para el \ac{LSH},
consiste en emplear diversas funciones de \textit{hash} para asociar
los elementos de un conjunto a distintas ‘casillas’.
De esta manera, los elementos similares entre sí,
tienen mayores probabilidades de ser agrupados en las mismas ‘casillas’,
que los demás elementos.
Luego, se considera como ‘par candidato’
a cualquier par de elementos que se encuentre en la misma ‘casilla’~\cite{Rajaraman:2011:MMD:2124405}.

Una forma efectiva de elegir las funciones de \textit{hash},
es tomar las $n$ firmas generadas mediante el \textit{MinHash}
y dividir la matriz en $b$ bandas que consten de $f$ filas cada una.
Para cada banda, hay una función de \textit{hash} que toma vectores de $f$ elementos
(parte de una columna dentro de esa banda)
y asocia la firma correspondiente a una ‘casilla’.

%Se puede usar la misma función de \textit{hash} para todas las bandas,
%pero se usa una lista de ‘casillas’ distinta para cada banda.
%De esta manera, las bandas no se mezclarán.
%las columnas con el mismo vector en diferentes

La cantidad de $b$ bandas y $f$ filas configura un umbral,
que determina si un elemento se vuelve un ‘par candidato’ o no.
Los elementos cuyo coeficiente de similitud se encuentra por encima de este umbral,
tienen una alta probabilidad de convertirse en pares candidatos;
caso contrario, aquellos elementos cuyo coeficiente de similitud se encuentra por debajo de este umbral,
tienen una baja probabilidad de convertirse en pares candidatos.

Una aproximación al umbral~\cite{Rajaraman:2011:MMD:2124405}, se puede obtener mediante la siguiente formula:

\begin{equation*}
 (\frac{1}{b})^{\frac{1}{r}}
\end{equation*}

Por ejemplo, si dividimos una matriz de firmas de $9$ filas en $3$ bandas,
obtenemos la Figura \ref{tab:conjuntoejemplo3}.
Donde se observa que la segunda y la cuarta columna mostradas explícitamente,
contienen el vector columna $(0, 2, 1)$,
lo que indica que ambas firmas serán agrupadas en la misma ‘casilla’.
Por lo tanto, este par de columnas será un ‘par candidato’.

\begin{figure}[h]
\small
\centering
\begin{tabular}{c|cccccccccccccccccccc|}
\cline{2-21}
			& & & & & & 1 & 0 & 0 & 0 & 2 & & & & & & & & & & \\
banda 1	    & \multicolumn{5}{c}{\textbf{...}} & 3 & 2 & 1 & 2 & 2 & \multicolumn{5}{c}{\textbf{...}} & & & & & \\
			& & & & & & 0 & 1 & 3 & 1 & 1 & & & & & & & & & & \\
\cline{2-21}
			& & & & & & & & & & & & & & & & & & & & \\
banda 2	    & & & & & & & & \textbf{\vdots} & & & & & & & & & & & & \\
			& & & & & & & & & & & & & & & & & & & & \\
\cline{2-21}
			& & & & & & & & & & & & & & & & & & & & \\
banda 3     & & & & & & & & & & & & & & & & & & & & \\
			& & & & & & & & & & & & & & & & & & & & \\
\cline{2-21}
\end{tabular}
\caption{Matriz de firmas.}
\label{tab:conjuntoejemplo3}
\end{figure}