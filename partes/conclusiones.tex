\chapter*{Conclusiones y Recomendaciones}

En el presente Proyecto de Grado se propuso la creación de un Sistema de Recomendación,
que se apoye sobre algoritmos de minería de datos para proveer sugerencias en tiempo real.
Estas sugerencias se realizan en base al fragmento de código provisto por el usuario en lenguaje c y c++,
el cual se hace coincidir con fragmentos de código que se encuentran en los tópicos de \textit{Stack Overflow}.

El sistema implementado demostró funcionar correctamente al sugerir tópicos,
cuyos fragmentos de código, son estructuralmente iguales o similares al proporcionado por el usuario.

Se encontraron los siguientes resultados parciales en la ejecución del presente proyecto:

\begin{itemize}
  \item El índice desarrollado en este proyecto es eficaz al obtener fragmentos de código que se encuentren previamente en la base de datos.
  \item Los fragmentos de código con baja especificidad tienen una mayor incidencia en los tópicos,
  lo que permite encontrar otros fragmentos iguales o similares estructuralmente con mayor probabilidad.
  Se encontró que $8$ líneas de código en promedio ofrece los mejores resultados.
\end{itemize}

El enfoque innovador utilizado en este proyecto,
se basa en la similitud sintáctica de los fragmentos de código para realizar sugerencias;
siendo en esencia, una comparación estructural pura.

Sin embargo, se subraya el hecho de que cualquier par de fragmentos de código,
estructuralmente iguales, pueden no poseer el mismo comportamiento.
Esto ocurre generalmente cuando la especificidad de los fragmentos es baja.

La lista de tópicos desplegada por la herramienta puede ser configurada,
restringiendo el espacio de búsqueda y alineando los resultados a los intereses del usuario.

El usuario puede dar preferencia a la búsqueda de preguntas o respuestas.
Además, es posible limitar la búsqueda a las preguntas con respuesta aceptada o a las respuestas aceptadas.
Así como, ordenar los resultados de acuerdo a la reputación del tópico.

De los resultados obtenidos se derivan los siguientes aspectos para su posterior estudio:

\begin{itemize}
  \item Expandir el preprocesamiento de los datos para incluir fragmentos de código que no se encuentren en el formato estándar.
  \item Ampliar el análisis sintáctico incluyendo el uso de un \textit{parser}, con la finalidad de incrementar la precisión.
  \item Implementar una función de \textit{hash} en el índice, con la finalidad de acelerar el proceso de búsqueda.
  \item Desarrollar una clasificación automática de los resultados mediante filtros especializados que incluyan información del contexto.
  \item Investigar algoritmos que permitan evaluar la contención de conjuntos, de forma similar a como se evalúa la igualdad de conjuntos mediante el \textit{MinHash}.
\end{itemize}